%% bare_conf_compsoc.tex
%% V1.4a
%% 2014/09/17
%% by Michael Shell
%% See:
%% http://www.michaelshell.org/
%% for current contact information.
%%
%% This is a skeleton file demonstrating the use of IEEEtran.cls
%% (requires IEEEtran.cls version 1.8s or later) with an IEEE Computer
%% Society conference paper.
%%
%% Support sites:
%% http://www.michaelshell.org/tex/ieeetran/
%% http://www.ctan.org/tex-archive/macros/latex/contrib/IEEEtran/
%% and
%% http://www.ieee.org/

%%*************************************************************************
%% Legal Notice:
%% This code is offered as-is without any warranty either expressed or
%% implied; without even the implied warranty of MERCHANTABILITY or
%% FITNESS FOR A PARTICULAR PURPOSE! 
%% User assumes all risk.
%% In no event shall IEEE or any contributor to this code be liable for
%% any damages or losses, including, but not limited to, incidental,
%% consequential, or any other damages, resulting from the use or misuse
%% of any information contained here.
%%
%% All comments are the opinions of their respective authors and are not
%% necessarily endorsed by the IEEE.
%%
%% This work is distributed under the LaTeX Project Public License (LPPL)
%% ( http://www.latex-project.org/ ) version 1.3, and may be freely used,
%% distributed and modified. A copy of the LPPL, version 1.3, is included
%% in the base LaTeX documentation of all distributions of LaTeX released
%% 2003/12/01 or later.
%% Retain all contribution notices and credits.
%% ** Modified files should be clearly indicated as such, including  **
%% ** renaming them and changing author support contact information. **
%%
%% File list of work: IEEEtran.cls, IEEEtran_HOWTO.pdf, bare_adv.tex,
%%                    bare_conf.tex, bare_jrnl.tex, bare_conf_compsoc.tex,
%%                    bare_jrnl_compsoc.tex, bare_jrnl_transmag.tex
%%*************************************************************************


% *** Authors should verify (and, if needed, correct) their LaTeX system  ***
% *** with the testflow diagnostic prior to trusting their LaTeX platform ***
% *** with production work. IEEE's font choices and paper sizes can       ***
% *** trigger bugs that do not appear when using other class files.       ***                          ***
% The testflow support page is at:
% http://www.michaelshell.org/tex/testflow/



\documentclass[conference,compsoc]{IEEEtran}
% Some/most Computer Society conferences require the compsoc mode option,
% but others may want the standard conference format.
%
% If IEEEtran.cls has not been installed into the LaTeX system files,
% manually specify the path to it like:
% \documentclass[conference,compsoc]{../sty/IEEEtran}





% Some very useful LaTeX packages include:
% (uncomment the ones you want to load)


% *** MISC UTILITY PACKAGES ***
%
%\usepackage{ifpdf}
% Heiko Oberdiek's ifpdf.sty is very useful if you need conditional
% compilation based on whether the output is pdf or dvi.
% usage:
% \ifpdf
%   % pdf code
% \else
%   % dvi code
% \fi
% The latest version of ifpdf.sty can be obtained from:
% http://www.ctan.org/tex-archive/macros/latex/contrib/oberdiek/
% Also, note that IEEEtran.cls V1.7 and later provides a builtin
% \ifCLASSINFOpdf conditional that works the same way.
% When switching from latex to pdflatex and vice-versa, the compiler may
% have to be run twice to clear warning/error messages.






% *** CITATION PACKAGES ***
%
\ifCLASSOPTIONcompsoc
  % IEEE Computer Society needs nocompress option
  % requires cite.sty v4.0 or later (November 2003)
%  \usepackage[nocompress]{cite}
\else
  % normal IEEE
  
\fi
% cite.sty was written by Donald Arseneau
% V1.6 and later of IEEEtran pre-defines the format of the cite.sty package
% \citet{} output to follow that of IEEE. Loading the cite package will
% result in citation numbers being automatically sorted and properly
% "compressed/ranged". e.g., [1], [9], [2], [7], [5], [6] without using
% cite.sty will become [1], [2], [5]--[7], [9] using cite.sty. cite.sty's
% \citet will automatically add leading space, if needed. Use cite.sty's
% noadjust option (cite.sty V3.8 and later) if you want to turn this off
% such as if a citation ever needs to be enclosed in parenthesis.
% cite.sty is already installed on most LaTeX systems. Be sure and use
% version 5.0 (2009-03-20) and later if using hyperref.sty.
% The latest version can be obtained at:
% http://www.ctan.org/tex-archive/macros/latex/contrib/cite/
% The documentation is contained in the cite.sty file itself.
%
% Note that some packages require special options to format as the Computer
% Society requires. In particular, Computer Society  papers do not use
% compressed citation ranges as is done in typical IEEE papers
% (e.g., [1]-[4]). Instead, they list every citation separately in order
% (e.g., [1], [2], [3], [4]). To get the latter we need to load the cite
% package with the nocompress option which is supported by cite.sty v4.0
% and later.





% *** GRAPHICS RELATED PACKAGES ***
%
\ifCLASSINFOpdf
  % \usepackage[pdftex]{graphicx}
  % declare the path(s) where your graphic files are
  % \graphicspath{{../pdf/}{../jpeg/}}
  % and their extensions so you won't have to specify these with
  % every instance of \includegraphics
  % \DeclareGraphicsExtensions{.pdf,.jpeg,.png}
\else
  % or other class option (dvipsone, dvipdf, if not using dvips). graphicx
  % will default to the driver specified in the system graphics.cfg if no
  % driver is specified.
  % \usepackage[dvips]{graphicx}
  % declare the path(s) where your graphic files are
  % \graphicspath{{../eps/}}
  % and their extensions so you won't have to specify these with
  % every instance of \includegraphics
  % \DeclareGraphicsExtensions{.eps}
\fi
% graphicx was written by David Carlisle and Sebastian Rahtz. It is
% required if you want graphics, photos, etc. graphicx.sty is already
% installed on most LaTeX systems. The latest version and documentation
% can be obtained at: 
% http://www.ctan.org/tex-archive/macros/latex/required/graphics/
% Another good source of documentation is "Using Imported Graphics in
% LaTeX2e" by Keith Reckdahl which can be found at:
% http://www.ctan.org/tex-archive/info/epslatex/
%
% latex, and pdflatex in dvi mode, support graphics in encapsulated
% postscript (.eps) format. pdflatex in pdf mode supports graphics
% in .pdf, .jpeg, .png and .mps (metapost) formats. Users should ensure
% that all non-photo figures use a vector format (.eps, .pdf, .mps) and
% not a bitmapped formats (.jpeg, .png). IEEE frowns on bitmapped formats
% which can result in "jaggedy"/blurry rendering of lines and letters as
% well as large increases in file sizes.
%
% You can find documentation about the pdfTeX application at:
% http://www.tug.org/applications/pdftex





% *** MATH PACKAGES ***
%
%\usepackage[cmex10]{amsmath}
% A popular package from the American Mathematical Society that provides
% many useful and powerful commands for dealing with mathematics. If using
% it, be sure to load this package with the cmex10 option to ensure that
% only type 1 fonts will utilized at all point sizes. Without this option,
% it is possible that some math symbols, particularly those within
% footnotes, will be rendered in bitmap form which will result in a
% document that can not be IEEE Xplore compliant!
%
% Also, note that the amsmath package sets \interdisplaylinepenalty to 10000
% thus preventing page breaks from occurring within multiline equations. Use:
%\interdisplaylinepenalty=2500
% after loading amsmath to restore such page breaks as IEEEtran.cls normally
% does. amsmath.sty is already installed on most LaTeX systems. The latest
% version and documentation can be obtained at:
% http://www.ctan.org/tex-archive/macros/latex/required/amslatex/math/





% *** SPECIALIZED LIST PACKAGES ***
%
%\usepackage{algorithmic}
% algorithmic.sty was written by Peter Williams and Rogerio Brito.
% This package provides an algorithmic environment fo describing algorithms.
% You can use the algorithmic environment in-text or within a figure
% environment to provide for a floating algorithm. Do NOT use the algorithm
% floating environment provided by algorithm.sty (by the same authors) or
% algorithm2e.sty (by Christophe Fiorio) as IEEE does not use dedicated
% algorithm float types and packages that provide these will not provide
% correct IEEE style captions. The latest version and documentation of
% algorithmic.sty can be obtained at:
% http://www.ctan.org/tex-archive/macros/latex/contrib/algorithms/
% There is also a support site at:
% http://algorithms.berlios.de/index.html
% Also of interest may be the (relatively newer and more customizable)
% algorithmicx.sty package by Szasz Janos:
% http://www.ctan.org/tex-archive/macros/latex/contrib/algorithmicx/




% *** ALIGNMENT PACKAGES ***
%
%\usepackage{array}
% Frank Mittelbach's and David Carlisle's array.sty patches and improves
% the standard LaTeX2e array and tabular environments to provide better
% appearance and additional user controls. As the default LaTeX2e table
% generation code is lacking to the point of almost being broken with
% respect to the quality of the end results, all users are strongly
% advised to use an enhanced (at the very least that provided by array.sty)
% set of table tools. array.sty is already installed on most systems. The
% latest version and documentation can be obtained at:
% http://www.ctan.org/tex-archive/macros/latex/required/tools/


% IEEEtran contains the IEEEeqnarray family of commands that can be used to
% generate multiline equations as well as matrices, tables, etc., of high
% quality.




% *** SUBFIGURE PACKAGES ***
%\ifCLASSOPTIONcompsoc
%  \usepackage[caption=false,font=footnotesize,labelfont=sf,textfont=sf]{subfig}
%\else
%  \usepackage[caption=false,font=footnotesize]{subfig}
%\fi
% subfig.sty, written by Steven Douglas Cochran, is the modern replacement
% for subfigure.sty, the latter of which is no longer maintained and is
% incompatible with some LaTeX packages including fixltx2e. However,
% subfig.sty requires and automatically loads Axel Sommerfeldt's caption.sty
% which will override IEEEtran.cls' handling of captions and this will result
% in non-IEEE style figure/table captions. To prevent this problem, be sure
% and invoke subfig.sty's "caption=false" package option (available since
% subfig.sty version 1.3, 2005/06/28) as this is will preserve IEEEtran.cls
% handling of captions.
% Note that the Computer Society format requires a sans serif font rather
% than the serif font used in traditional IEEE formatting and thus the need
% to invoke different subfig.sty package options depending on whether
% compsoc mode has been enabled.
%
% The latest version and documentation of subfig.sty can be obtained at:
% http://www.ctan.org/tex-archive/macros/latex/contrib/subfig/




% *** FLOAT PACKAGES ***
%
%\usepackage{fixltx2e}
% fixltx2e, the successor to the earlier fix2col.sty, was written by
% Frank Mittelbach and David Carlisle. This package corrects a few problems
% in the LaTeX2e kernel, the most notable of which is that in current
% LaTeX2e releases, the ordering of single and double column floats is not
% guaranteed to be preserved. Thus, an unpatched LaTeX2e can allow a
% single column figure to be placed prior to an earlier double column
% figure. The latest version and documentation can be found at:
% http://www.ctan.org/tex-archive/macros/latex/base/


%\usepackage{stfloats}
% stfloats.sty was written by Sigitas Tolusis. This package gives LaTeX2e
% the ability to do double column floats at the bottom of the page as well
% as the top. (e.g., "\begin{figure*}[!b]" is not normally possible in
% LaTeX2e). It also provides a command:
%\fnbelowfloat
% to enable the placement of footnotes below bottom floats (the standard
% LaTeX2e kernel puts them above bottom floats). This is an invasive package
% which rewrites many portions of the LaTeX2e float routines. It may not work
% with other packages that modify the LaTeX2e float routines. The latest
% version and documentation can be obtained at:
% http://www.ctan.org/tex-archive/macros/latex/contrib/sttools/
% Do not use the stfloats baselinefloat ability as IEEE does not allow
% \baselineskip to stretch. Authors submitting work to the IEEE should note
% that IEEE rarely uses double column equations and that authors should try
% to avoid such use. Do not be tempted to use the cuted.sty or midfloat.sty
% packages (also by Sigitas Tolusis) as IEEE does not format its papers in
% such ways.
% Do not attempt to use stfloats with fixltx2e as they are incompatible.
% Instead, use Morten Hogholm'a dblfloatfix which combines the features
% of both fixltx2e and stfloats:
%
% \usepackage{dblfloatfix}
% The latest version can be found at:
% http://www.ctan.org/tex-archive/macros/latex/contrib/dblfloatfix/




% *** PDF, URL AND HYPERLINK PACKAGES ***
%
%\usepackage{url}
% url.sty was written by Donald Arseneau. It provides better support for
% handling and breaking URLs. url.sty is already installed on most LaTeX
% systems. The latest version and documentation can be obtained at:
% http://www.ctan.org/tex-archive/macros/latex/contrib/url/
% Basically, \url{my_url_here}.




% *** Do not adjust lengths that control margins, column widths, etc. ***
% *** Do not use packages that alter fonts (such as pslatex).         ***
% There should be no need to do such things with IEEEtran.cls V1.6 and later.
% (Unless specifically asked to do so by the journal or conference you plan
% to submit to, of course. )

% Packages used for adding comments
\usepackage[portuguese]{babel}
\selectlanguage{portuguese}
\usepackage[T1]{fontenc}
\usepackage[utf8]{inputenc}
\usepackage[colorlinks]{hyperref}
\usepackage[colorinlistoftodos]{todonotes}
\usepackage[numbers]{natbib} % para \citetauthor{} e outros
\usepackage{booktabs}
\usepackage{multirow}
\usepackage{graphicx}


% correct bad hyphenation here
\hyphenation{op-tical net-works semi-conduc-tor}


\begin{document}
%
% paper title
% Titles are generally capitalized except for words such as a, an, and, as,
% at, but, by, for, in, nor, of, on, or, the, to and up, which are usually
% not capitalized unless they are the first or last word of the title.
% Linebreaks \\ can be used within to get better formatting as desired.
% Do not put math or special symbols in the title.
\title{Análise da Correlação de Circutos Integrados por meio de\\ \textit{Clustering} de Acordo com Volume de Importações}

% author names and affiliations
% use a multiple column layout for up to three different
% affiliations
\author{\IEEEauthorblockN{Sandro Leite Furtado}
\IEEEauthorblockA{Mestrado em Computação Aplicada\\
Universidade de Brasília\\
Brasília DF, 70910-900, Brasil\\
Email: sandroleitefurtado@gmail.com}
\and
\IEEEauthorblockN{Tiago Pereira Vidigal}
\IEEEauthorblockA{Mestrado em Computação Aplicada\\
Universidade de Brasília\\
Brasília DF, 70910-900, Brasil\\
Email: tiago.vidigal@aluno.unb.br}
\and
\IEEEauthorblockN{William Oliveira Camelo}
\IEEEauthorblockA{Mestrado em Computação Aplicada\\
Universidade de Brasília\\
Brasília DF, 70910-900, Brasil\\
Email: }
}


% conference papers do not typically use \thanks and this command
% is locked out in conference mode. If really needed, such as for
% the acknowledgment of grants, issue a \IEEEoverridecommandlockouts
% after \documentclass

% for over three affiliations, or if they all won't fit within the width
% of the page (and note that there is less available width in this regard for
% compsoc conferences compared to traditional conferences), use this
% alternative format:
% 
%\author{\IEEEauthorblockN{Michael Shell\IEEEauthorrefmark{1},
%Homer Simpson\IEEEauthorrefmark{2},
%James Kirk\IEEEauthorrefmark{3}, 
%Montgomery Scott\IEEEauthorrefmark{3} and
%Eldon Tyrell\IEEEauthorrefmark{4}}
%\IEEEauthorblockA{\IEEEauthorrefmark{1}School of Electrical and Computer Engineering\\
%Georgia Institute of Technology,
%Atlanta, Georgia 30332--0250\\ Email: see http://www.michaelshell.org/contact.html}
%\IEEEauthorblockA{\IEEEauthorrefmark{2}Twentieth Century Fox, Springfield, USA\\
%Email: homer@thesimpsons.com}
%\IEEEauthorblockA{\IEEEauthorrefmark{3}Starfleet Academy, San Francisco, California 96678-2391\\
%Telephone: (800) 555--1212, Fax: (888) 555--1212}
%\IEEEauthorblockA{\IEEEauthorrefmark{4}Tyrell Inc., 123 Replicant Street, Los Angeles, California 90210--4321}}




% use for special paper notices
%\IEEEspecialpapernotice{(Invited Paper)}




% make the title area
\maketitle

% As a general rule, do not put math, special symbols or citations
% in the abstract

\begin{abstract}
The abstract goes here.
\end{abstract}


% keywords




% For peer review papers, you can put extra information on the cover
% page as needed:
% \ifCLASSOPTIONpeerreview
% \begin{center} \bfseries EDICS Category: 3-BBND \end{center}
% \fi
%
% For peerreview papers, this IEEEtran command inserts a page break and
% creates the second title. It will be ignored for other modes.
\IEEEpeerreviewmaketitle



\section{Introdução} \label{sec_intro}
% no \IEEEPARstart
% You must have at least 2 lines in the paragraph with the drop letter
% (should never be an issue)

%\subsection{Subsection Heading Here}
%Subsection text here.
%\subsubsection{Subsubsection Heading Here}
%Subsubsection text here.

% An example of a floating figure using the graphicx package.
% Note that \label must occur AFTER (or within) \caption.
% For figures, \caption should occur after the \includegraphics.
% Note that IEEEtran v1.7 and later has special internal code that
% is designed to preserve the operation of \label within \caption
% even when the captionsoff option is in effect. However, because
% of issues like this, it may be the safest practice to put all your
% \label just after \caption rather than within \caption{}.
%
% Reminder: the "draftcls" or "draftclsnofoot", not "draft", class
% option should be used if it is desired that the figures are to be
% displayed while in draft mode.
%
%\begin{figure}[!t]
%\centering
%\includegraphics[width=2.5in]{myfigure}
% where an .eps filename suffix will be assumed under latex, 
% and a .pdf suffix will be assumed for pdflatex; or what has been declared
% via \DeclareGraphicsExtensions.
%\caption{Simulation results for the network.}
%\label{fig_sim}
%\end{figure}
% Note that IEEE typically puts floats only at the top, even when this
% results in a large percentage of a column being occupied by floats.
% An example of a double column floating figure using two subfigures.
% (The subfig.sty package must be loaded for this to work.)
% The subfigure \label commands are set within each subfloat command,
% and the \label for the overall figure must come after \caption.
% \hfil is used as a separator to get equal spacing.
% Watch out that the combined width of all the subfigures on a 
% line do not exceed the text width or a line break will occur.
%
%\begin{figure*}[!t]
%\centering
%\subfloat[Case I]{\includegraphics[width=2.5in]{box}%
%\label{fig_first_case}}
%\hfil
%\subfloat[Case II]{\includegraphics[width=2.5in]{box}%
%\label{fig_second_case}}
%\caption{Simulation results for the network.}
%\label{fig_sim}
%\end{figure*}
%
% Note that often IEEE papers with subfigures do not employ subfigure
% captions (using the optional argument to \subfloat[]), but instead will
% reference/describe all of them (a), (b), etc., within the main caption.
% Be aware that for subfig.sty to generate the (a), (b), etc., subfigure
% labels, the optional argument to \subfloat must be present. If a
% subcaption is not desired, just leave its contents blank,
% e.g., \subfloat[].
% An example of a floating table. Note that, for IEEE style tables, the
% \caption command should come BEFORE the table and, given that table
% captions serve much like titles, are usually capitalized except for words
% such as a, an, and, as, at, but, by, for, in, nor, of, on, or, the, to
% and up, which are usually not capitalized unless they are the first or
% last word of the caption. Table text will default to \footnotesize as
% IEEE normally uses this smaller font for tables.
% The \label must come after \caption as always.
%
%\begin{table}[!t]
%% increase table row spacing, adjust to taste
%\renewcommand{\arraystretch}{1.3}
% if using array.sty, it might be a good idea to tweak the value of
% \extrarowheight as needed to properly center the text within the cells
%\caption{An Example of a Table}
%\label{table_example}
%\centering
%% Some packages, such as MDW tools, offer better commands for making tables
%% than the plain LaTeX2e tabular which is used here.
%\begin{tabular}{|c||c|}
%\hline
%One & Two\\
%\hline
%Three & Four\\
%\hline
%\end{tabular}
%\end{table}
% Note that the IEEE does not put floats in the very first column
% - or typically anywhere on the first page for that matter. Also,
% in-text middle ("here") positioning is typically not used, but it
% is allowed and encouraged for Computer Society conferences (but
% not Computer Society journals). Most IEEE journals/conferences use
% top floats exclusively. 
% Note that, LaTeX2e, unlike IEEE journals/conferences, places
% footnotes above bottom floats. This can be corrected via the
% \fnbelowfloat command of the stfloats package.
A facilidade de se coletar e armazenar dados gerou a necessidade de analisá-los, de acordo com \citet{han_data_2011}. A mineração de dados é o processo de descoberta de padrões significativos de dados e pode ser usado para essa análise como destaca \citet{witten_data_2005}. Essa abordagem permite o levantamento de informações relevantes para tomadores de decisão fazerem escolhas baseadas em dados.

Chipus Microeletrônica é uma empresa privada brasileira que atua como \textit{Design House} no setor de Semicondutores e possui o interesse de aumentar o número de projetos de produto, como o desenvolvimento de \textit{Application Specific Integrated Circuits} (ASICs). Esse interesse estratégico visa aumentar o potencial ganho constante que esse tipo de projeto fornece em contraste com projetos de serviço. A definição de quais sistemas possuem uma maior chance de sucesso é desafiadora, mas fornece informações valiosas para atingir os objetivos de negócio.

O aumento de projetos de produto pode ser viabilizado mapeando oportunidades de mercado para a plataforma ICX da Chipus, um conjunto configurável por camada de metal de \textit{Intellectual Properties} (IPs) digitais e analógicos \cite{datasheet_icx}. Um único produto capaz de juntar vários componentes clássicos de forma rápida e barata se torna muito chamativo para empresas importando múltiplos circuitos integrados para seus projetos. No entanto, sua efetividade depende que o ICX contenha os circuitos necessários para o cliente com o mínimo de excesso possível por questões de custo, área e consumo.

O objetivo deste trabalho é mapear grupos de \textit{Integrated Circuits} (ICs) que são comumente utilizados juntos. Isso pode ser estimado pelo volume de importações de cada circuito, informação amplamente disponível para o Brasil e outros países. A geração de \textit{clusters} de acordo com tendências de importação permitirá a definição de diferentes versões da plataforma ICX da Chipus que tenham maior chance de impacto no mercado.

A divisão do trabalho é conforme segue: na seção \ref{sec_rev} é apresentada a revisão da literatura sobre o tema apresentando conceitos utilizados na concepção e trabalhos correlatos. Na seção \ref{sec_metod} é apresentada a metodologia adotada para o desenvolvimento deste trabalho, a seção \ref{sec_result} apresenta os resultados encontrados e finaliza-se com a seção \ref{sec_concl} concluindo os achados.

\section{Revisão de Literatura}\label{sec_rev}
Essa seção apresentará a revisão de literatura do estado da arte relacionado ao escopo deste trabalho. Primeiro será apresentada uma visão geral sobre \textit{Data Mining}, em seguida sobre a técnica escolhida e por fim sobre o escopo negocial a ser analizado.

\subsection{\textit{Data Mining}}
\textit{Data Mining}, ou a mineração de dados em português, pode ser entendida como um processo de descoberta de informações estratégicas para predição de futuros comportamentos e, de acordo com \citet{witten_data_2005} é a extração implícita, previamente desconhecida, e potencialmente utilizável de informações sobre dados. 

Para uma empresa, a aprendizagem dos dados já existentes em bancos de dados, textos ou processos pode ser considerada vantajosa no âmbito da competitividade, de acordo \citet{thuraisingham_data_1998} uma vez que informações estratégicas, para tomada de decisão, sejam geradas.

\subsection{Clusterização}
A clusterização é uma técnica para classificação de dados que pode ser usada quando se tem um volume grande e pouco conhecimento sobre eles \cite{reviewcluster}. A técnica de mineração coloca dados relacionados ou homogêneos em grupos sem conhecimento avançado dos grupos, essa abordagem é conhecida como clássica (crisp), uma vez que seus elementos estão contidos em uma única classe. \cite{Rai2010cluster}.

\subsubsection{\textit{Time-Series}}
Segundo \citet{Antunes2001TemporalDM} uma sequencia temporal ou \textit{time-series}, em inglês, é um tipo de clusterização para uma sequência de elementos contínuos de valor real.

\subsubsection{\textit{Fuzzy}}
A abordagem fuzzy difere da abordagem clássica de clusterização, uma vez que um elemento pode pertencer a várias classes e assumir valores variados \cite{fuzzySC2002}.

Segundo \citet{bezdek_tsao_pal}, uma abordagem clássica, por vezes pode ser muito restritiva e inviável, uma vez que pode levar a imprecisão e incompletude dos dados. Essa imprecisão pode ocorrer por diferentes fatores, tais como, erros em instrumentos ou ruídos na amostra de dados podem levar, de forma parcial, a valores não confiáveis de determinados atributos.

Nesse sentido, é adequado o uso da Teoria de Conjuntos Fuzzy (TCF) para representar valores imprecisos. Nessa abordagem utilizam-se variáveis linguísticas e restrições para descrever os valores de atributos, ao contrário de fornecer uma representação numérica exata aos dados com valores incertos dos atributos \cite{fuzzySC2002}.

\subsubsection{\textit{Hierarchical clustering}}

Essa abordagem atua agrupando atributos de dados (séries temporais) em uma árvore de clusters. São diferenciados dois tipos de métodos de agrupamento hierárquico, sendo eles: aglomerativo e divisivo. 

O método de agrupamento hierárquico aglomerativo atribui a cada objeto um cluster próprio, a partir disso, mescla esses clusters menores em clusters cada vez maiores, de modo a se criar um cluster ainda maior e que contenha a representação de todos os objetos e as condições sejam satisfeitas. Cabe destacar que o método aglomerativo é mais usado que o divisivo em geral.

\subsection{Circuitos Integrados}

Circuito integrado é um conjunto de componentes ativos e passivos interconectados para implementar um sistema eletrônico em escala micrométrica. O crescimento de complexidade destes sistemas tem crescido de forma acelerada ao longo dos anos, requerendo modularização e integração de blocos de circuito específicos. Isso se assemelha ao que ocorreu com sistemas em software.

A modularização de circuitos proporcionou o desenvolvimento de IPs, sistemas autocontidos que podem ser vendidos para reuso em sistemas distintos. A compra de IPs prontos ao invés do desenvolvimento é uma recorrente tomada de decisão a ser feita em projetos de hardware dado o custo de desenvolvimento e manufatura. Usualmente, múltiplos blocos independentes são integrados para cada produto, potencialmente requerendo lidar com diferentes fornecedores.

\subsection{CRISP-DM}
A mineração de dados é um processo interativo e iterativo em que muitos passos precisam ser repetidamente refinados, a fim de proporcionar uma solução adequada para o problema de análise de dados, de acordo com \citet{wang_data_2008}. O padrão de mineração utilizado neste trabalho será o CRISP-DM (\textit{Cross Industry Standard Process for Data Mining}). Este modelo define um processo que reflete o ciclo de vida de um projeto de mineração, formado por seis fases, são elas: entendimento do negócio, além do entendimento, preparação e modelagem dos dados, finalizando como sua avaliação e implementação. \citet{kononenko_machine_2007} identifica que os relacionamentos entre essas fases são ciclicamente iterativas até que algum objetivo desejado seja alcançado. 

O entendimento do negócio visa o esclarecimento do contexto, objetivos comerciais da empresa e objetivos da mineração de dados. O entendimento dos dados consiste em uma primeira coleta e avaliação do dado disponível, seguido pela sua preparação que consiste em selecionar, limpar, construir e integrar. Por fim, a modelagem e seu respectivo teste permite a avaliação e revisão dos resultados, embasando o processo de implementação para transformar as descobertas obtidas em ações de melhoria para a empresa \cite{guia_crisp_dm}.

\subsection{Trabalhos Correlatos}

\citet{DUrso2013} abordou a classificação de séries temporais utilizando fuzzy, no contexto financeiro. Neste trabalho foram propostos dois modelos de agrupamento fuzzy baseados em modelos GARCH. Na adequação das medidas de distância para o primeiro modelo \citet{DUrso2013} utilizou a métrica autorregressiva clássica. O segundo utilizou a distância de Caiado, uma distância do tipo Mahalanobis, baseada em parâmetros GARCH estimados e covariâncias que levaram em consideração as informações sobre a estrutura de volatilidade das séries temporais. O trabalho apresenta uma aplicação ao problema de classificação de 29 séries temporais de taxas de câmbio do euro em relação às moedas internacionais e identificou três clusters de taxas de câmbio do Euro, caracterizados por diferentes níveis de estabilidade em termos de flutuações da volatilidade condicional. O trabalhou concluiu que existe uma sugestão de que o uso da distância de Caiado ajuda a descobrir a imprecisão da estrutura do grupo. Os resultados do trabalho mostraram que os clusterings propostos são capazes de revelar características importantes das séries temporais analisadas e detectaram padrões ocultos em grandes amostras de séries temporais financeiras.



\section{Metodologia}\label{sec_metod}

A metodologia seguida neste trabalho é baseada no \textit{CRISM-DM} \cite{guia_crisp_dm}. O entendimento de negócios resumido foi consolidado na seção \ref{sec_intro}, enquanto as demais fases intermediárias são descritas nesta seção. A fase final de implementação foge do escopo deste artigo.

\subsection{Entendimento dos Dados}

Os dados de importação iniciais estão disponíveis na base de dados da Receita Federal do Brasil. Os campos dos dados obtidos são listados e uma primeira exploração deles levantar as primeiras impressões. O mapeando problemas de formatação e inconsistências a serem tratadas deve ser realizado e registrado para guiar a preparação dos dados.

Importações de outros países podem fornecer informações interessantes, tornando valioso o entendimento desses dados. Demais países do Mercosul são candidatos naturais dada as similaridades, porém integrantes da União Europeia (EU) e dos Estados Unidos (USA) também podem ser usados. A diversificação de países auxilia na redução de particularidades nacionais e aumenta a quantidade de dados que podem ser usados na mineração.

\subsection{Preparação dos Dados}

Circuitos integrados em trocas internacionais são registrados por códigos de prefixo 8542 como definido pelo padrão de nomenclatura internacional \textit{Harmonized System} (HS) \cite{standard_hs}. Esse código pode ser usado para selecionar apenas os registros de interesse dos dados de importação. Essa filtragem pode ser repetida para todos os países sendo analisados.

Os dados selecionados são limpos para resolver quaisquer problemas de qualidade levantados e nenhum dado novo precisará ser construído uma vez que apenas o volume de importações será analisado. A integração dos dados será necessária caso múltiplos países sejam analisados, requerendo padronizar o nome dos campos e a formatação dos dados nos registros.

Os países, apesar de seguirem o padrão HS, geralmente possuem uma nomenclatura derivada com subcategorias para maior especificidade dos produtos. Países do Mercosul, como o Brasil, seguem a Nomenclatura Comum do Mercosul (NCM) \cite{standard_ncm}, países da EU pelo padrão \textit{Combined Nomenclature} (CN) \cite{standard_cn} e USA pelo padrão \textit{Schedule B} \cite{standard_scheduleb}. Uma vez que as subcategorias entre esses vários padrões não é correspondente, a modelagem deve ser feita por conjuntos de países com nomenclaturas idênticas.

\subsection{Modelagem}

Os grupos de circuitos integrados mais utilizados juntos podem ser determinados com as tendências de importação, que é uma série temporal de dados de mesmo tamanho. Estes componentes não precisam ser exclusivos de apenas um grupo, uma vez que é comum circuitos clássicos estarem presentes em várias aplicações distintas. Com isso, \textit{fuzzy time-series clustering} é uma modelagem compatível com a análise proposta \cite{reviewcluster}.

O teste dos resultados do modelo serão feitos com um subconjunto dos dados tratados. O critério de excelência é definido pela taxa de erro do modelo usando métodos de medida de similaridade para geração de métricas simples e estáveis \cite{reviewcluster}. A distancia euclidiana é uma solução clássica e pode ser usada diretamente, mas outro métodos podem ser associados para maior precisão, como correlação cruzada \cite{liaosurvey}.

\subsection{Avaliação}

As iterações com diferentes configurações do modelo e algoritmos cria diferentes conjuntos de soluções, cada um com um número distinto de \textit{clusters} e composição de circuitos. Essas informações levantadas podem gerar descobertas que devem ser registradas.

As soluções e descobertas devem ser avaliadas considerando os objetivos de negócios junto a um especialista. Uma revisão dessas informações evidencia êxitos e levanta pontos a serem melhorados, podendo inclusive gerar uma nova iteração da mineração a fim de refinar dados e/ou modelo. A mineração pode ser concluída apenas ao atingir os objetivos de negócio.

\section{Resultados}\label{sec_result}

To be done.

\section{Conclusão} \label{sec_concl}

To be done.


% conference papers do not normally have an appendix



% trigger a \newpage just before the given reference
% number - used to balance the columns on the last page
% adjust value as needed - may need to be readjusted if
% the document is modified later
%\IEEEtriggeratref{8}
% The "triggered" command can be changed if desired:
%\IEEEtriggercmd{\enlargethispage{-5in}}

% referencia section

% can use a bibliography generated by BibTeX as a .bbl file
% BibTeX documentation can be easily obtained at:
% http://www.ctan.org/tex-archive/biblio/bibtex/contrib/doc/
% The IEEEtran BibTeX style support page is at:
% http://www.michaelshell.org/tex/ieeetran/bibtex/
%\bibliographystyle{IEEEtran}
% argument is your BibTeX string definitions and bibliography database(s)
%\bibliography{IEEEabrv,../bib/paper}
%
% <OR> manually copy in the resultant .bbl file
% set second argument of \begin to the number of references
% (used to reserve space for the reference number labels box)

\bibliographystyle{IEEEtranN}
\bibliography{referencias.bib}


% that's all folks
\end{document}
